%% LyX 2.3.6.1 created this file.  For more info, see http://www.lyx.org/.
%% Do not edit unless you really know what you are doing.
\documentclass[12pt,english]{extarticle}
\usepackage{mathptmx}
\renewcommand{\familydefault}{\rmdefault}
\usepackage[T1]{fontenc}
\usepackage[utf8]{inputenc}
\usepackage{geometry}
\geometry{verbose,tmargin=2cm,bmargin=2cm,lmargin=2.5cm,rmargin=2.5cm}
\usepackage{color}
\usepackage{babel}
\usepackage{float}
\usepackage[unicode=true,pdfusetitle,
 bookmarks=true,bookmarksnumbered=false,bookmarksopen=false,
 breaklinks=false,pdfborder={0 0 0},pdfborderstyle={},backref=false,colorlinks=true]
 {hyperref}
\hypersetup{
 linkcolor = blue,urlcolor  = pink,citecolor = red,anchorcolor = green}

\makeatletter
%%%%%%%%%%%%%%%%%%%%%%%%%%%%%% User specified LaTeX commands.
\date{}
\hypersetup{urlcolor=blue}
\usepackage[utf8]{inputenc}
\usepackage[T1]{fontenc}

\usepackage{xcolor}

%\usepackage{natbib}
%\usepackage[authoryear-comp]{natbib}
%\usepackage{breqn}
%linkcolor = black,urlcolor  = black,citecolor = black,anchorcolor = black
%linkcolor = blue,urlcolor  = pink,citecolor = red,anchorcolor = green

\renewcommand\thesection{\arabic{section}.}
\renewcommand\thesubsection{\thesection\arabic{subsection}.}
\renewcommand\thesubsubsection{\thesubsection\arabic{subsubsection}.}

%\linespread{1.25}
\usepackage{cite}





\usepackage[noabbrev,capitalise]{cleveref}
%\crefname{figure}{figure}{figures}
%\Crefname{figure}{Figure}{Figures}
\crefname{figure}{Fig.}{Figs.}
\Crefname{figure}{Fig.}{Figs.}
%\crefname{table}{table}{tables}
%\Crefname{table}{Table}{Tables}
\crefname{table}{Table}{Tables}
\Crefname{table}{Table}{Tables}
%\crefname{equation}{eq.}{eqs.}
%\Crefname{equation}{Eq.}{Eqs.}
\crefname{equation}{Eq.}{Eqs.}
\Crefname{equation}{Eq.}{Eqs.}
\crefname{section}{Section}{Sections}
\Crefname{section}{Section}{Sections}

\newcommand{\crefrangeconjunction}{--}

\AtBeginDocument{%
\let\ref\cref
}

\AtBeginDocument{%
\let\Ref\Cref
}

\makeatother

\begin{document}
\title{Connecting .... 4 Aerospace PhD Students}

\maketitle
\textbf{P}PP (\textbf{Papers}, PhD students, Projects)

These are three \textbf{P} which are important for Assitant/Associate/Full
Professor.
\begin{itemize}
\item Read \textbf{\href{https://www.ugc.ac.in/pdfnews/5323630_New_Draft_UGCRegulation-2018-9-2.pdf}{ugc guidelines for assistant professor}}
to get the idea what Indian colleges are looking in the candidate.
\item However, for PhD student, who will seek for assistant professor position
in their future, \textbf{only} papers are important. 
\item PhD topic must be such that PhD candidate can publish papers, get
postdoc, get job, and get project easily. In other words, research
area should be new and trending research topic which make you future
life easy and peaceful.
\end{itemize}

\part{PhD}
\begin{itemize}
\item Regularly visit \href{http://www.apna.iitkgp.ac.in/}{Apna IITKGP}
and attend relevant seminar (apart from our department).
\item Calendar for tracking the conference and targets. Make use of holiday
and Bdays to remain connected with friends/Prof./Guide/Foreign Prof. 
\item \textbf{Sticky Notes} for scheduling the tasks
\item Make your website and keep it update. Use GitHub
\item List of professor 
\begin{itemize}
\item Make a list of professor in your area so that you can find out who
are the people working in your area.
\item Helpful in targeted collaboration
\item Helpful during synopsis
\item Helpful after PhD
\item Helpful in making the list of university
\item Subscribe to their google scholar profile, RG profile, and other profiles.
\end{itemize}
\item List of University
\begin{itemize}
\item Make a list of University in your area so that you can find apply
for postDoc and find the postion and fellowship
\end{itemize}
\item List of Fellowship
\begin{itemize}
\item \href{https://international.iitkgp.ac.in/programs/}{DUO-India Fellowship Programme }
\end{itemize}
\item List of Journal and \textbf{make alert }\textcolor{red}{very important}
\item Project (very important): learn how to write project and how to find
the funding agency
\item Article
\begin{itemize}
\item Review article will give citation.
\item An article has 4 point; 2 point for first author (you); 2 points for
remaining author;
\item Include your friends (contributor) name as second author; First author
(you) still getting 2 points; your friend (contributor) will be benefit.
\item If you contribute in your friends paper, your friend as first author
will get 2 points, you as second author will get some points, and
rest authors (supervisor/guide)will get some points 
\item Always be a first author or second author or corresponding author.
\item Check the quality of journal at \href{https://www.scimagojr.com}{Scimago}
\end{itemize}
\item Book chapter (important): 1 and 2 are sufficient
\item Patent (optional): 1 is sufficient
\item Lab Setup: learn the experience, it will be helpful after getting
the Asst. Prof. position. Take certificate very important.
\item Website for searching the literature
\begin{enumerate}
\item Scopus and click on number in the \textbf{cited by }column.
\item Research Gate
\item Journal website very helpful; while put latest paper for target journal.
\begin{enumerate}
\item Always cite last 2 year papers of targeted journal. For editor, his/her
journal impact is important and you are helping to improve the impact
factor.
\end{enumerate}
\end{enumerate}
\item Citation key for bibTEX with DOI
\begin{enumerate}
\item Crossref \href{https://search.crossref.org/}{click here}
\item Google scholar does not include DOI
\item Use cite option for particular article from the journal website
\end{enumerate}
\item Notemaking: \textbf{Evernote}
\item Cloud to keep the codes and reports
\begin{enumerate}
\item DropBox; Signup using this link and get 500 MB \href{https://www.dropbox.com/referrals/AADZBivDNz1lITBEAyy5gB0tqwprSbAGwHk?src=global9}{click here}
\item Mega
\item OneDrive use iitkgp email for 1TB
\item GoogleDrive
\end{enumerate}
\item Everything for searching the files and other things
\item WordWeb, WordTune, Grammarly, \href{https://quillbot.com/}{quillbot}
for writing papers
\item Writing report/papers: MikTEX + TeXStudio + LyX
\begin{enumerate}
\item Write paper/report in LyX to save the time then export the tex file 
\item Run the tex file in the texstudio and submit the zip file in journal
\item Texmaker is also good instead of TexStudio
\item \href{https://mathpix.com/}{mathpix} for writing equation from image.
\end{enumerate}
\item For presentation: Beamer document in LyX
\item Texteditor: Notepad++ or sublime text
\item Drawings: Inkscape (free) or Illustrator
\begin{itemize}
\item First install miktex
\item Then, integrate \href{https://textext.github.io/textext/}{textext}
in Inkscape for writing latex formula.
\end{itemize}
\item Version control for code: Git + GitHuB
\item Compression and extraction: 7Zip (winRar give popups)
\item To mount image/iso file: Virtual Clone Drive
\item Coding
\begin{itemize}
\item MATLAB, Mathematica, Python
\item C, C++, Julia
\item Data structure and algorithm
\item Parallel computing, Open MP, MPI
\end{itemize}
\item Software
\begin{enumerate}
\item ANSYS
\item ABAQUS
\item COMSOL
\item NASTRAN \& PATRAN
\item other check \href{http://www.cic.iitkgp.ac.in/?q=node/28}{CIC website}
and \href{http://swrepo.iitkgp.ac.in/}{software repo}
\end{enumerate}
\item Plotting
\begin{enumerate}
\item matplotlib in python
\item MATLAB
\item Origin Pro, TecPlot, or GNU plot
\end{enumerate}
\item Computational power
\begin{itemize}
\item HPC
\item Paramshakti
\end{itemize}
\item Use Remote desktop or Teamviewer and work from room. (Find the video
inside the repo) 
\item Course/Workshop \href{https://www.noticebard.com/\%7C\%7CNoticeboard}{Noticeboard}
\begin{itemize}
\item Short term course
\item IIT KGP \href{http://iitkgp.ac.in/events}{Events} and \href{https://erp.iitkgp.ac.in/CEP/courses.htm}{courses}
\item \href{https://gian.iitkgp.ac.in/ccourses/approvecourses2}{GIAN}
\item Online course
\item Attend workshop organized by IIT KGP and other universities
\end{itemize}
\item Building your professional network. \textbf{very important}.
\item Make a list of professional bodies for prospect membership. 
\end{itemize}

\section{Semester 1}
\begin{enumerate}
\item Make a list of conference
\item Learn LyX/Latex for writing conference paper
\item Learn Inkscape
\item Try for PMRF fellowship
\item Attend our library workshop
\item Learn how to used HPC
\item Learn about journal writing or proposal writing or any other stuff.
\item Improve coding
\item Important websites: \href{https://international.iitkgp.ac.in/}{Internation relation IITKGP},
\href{http://www.cdc.iitkgp.ac.in/}{CDC}, \href{https://www.facebook.com/IRCIITKGP/}{Internation Relation Cell Facebook},
\href{https://www.facebook.com/aeroiitkgp/}{Department Facebook Page},
\href{http://www.apna.iitkgp.ac.in/}{APNA IITKGP} and check semester
6. To grab the opportunity, don't miss the information. 
\end{enumerate}

\section{Semester 2}
\begin{enumerate}
\item Read fundamental book and papers
\item Know all your seniors and their work areas. With this we can get some
help from seniors.
\item Do literature survey
\item Balance TA work 
\begin{enumerate}
\item There is no point for TA. Balance your research work.
\item But if you are under your guide/DSC then put effort. It will improve
your relation with guide/DSC.
\item If TA subject is within your research field then TA will help in improving
the fundamental.
\end{enumerate}
\item Work on comprehensive viva \href{https://docs.google.com/spreadsheets/d/13vbM4ww2g8GRpI5DYbnLKfR5EeMsKTbWfFFpLLtjDe4/edit\#gid=0}{Questions}
\end{enumerate}

\section{Semester 3}
\begin{enumerate}
\item Find the gaps in literature, discuss with group members.
\item Fix objective and scopes 
\item Work on Registration
\begin{enumerate}
\item Download relevant thesis from \href{http://www.idr.iitkgp.ac.in/xmlui/}{IDR-IITKGP}
\item Download thesis from other sources like \href{https://shodhganga.inflibnet.ac.in/}{shodhganga},
\href{https://www.proquest.com/}{proquest}, \href{https://oatd.org/}{OATD}
\end{enumerate}
\item Publish conference paper (in collaboration or individual). 
\begin{enumerate}
\item Conference paper has 1 point (only 0.5 for first author). There is
cap of maximum 5 points.
\item It will not help to become the Asst. Prof.
\item It will help to improve English and writing skill and other skills
require to write papers.
\item Conference is for making contacts only and getting new idea.
\begin{enumerate}
\item For Prof., thesis reviewer and collobration.
\end{enumerate}
\end{enumerate}
\end{enumerate}

\section{Semester 4}
\begin{enumerate}
\item Work on enhancement 
\item Search code from GitHub
\item Keep working on research objective for writing your first paper 
\item Start looking for short term course and other.
\begin{enumerate}
\item Make contact with that prof. during the short term course
\end{enumerate}
\item Write first paper (in collaboration or individual): let it be rejected,
feedback is very important to improve.
\end{enumerate}

\section{Semester 5}
\begin{enumerate}
\item Start searching for prof. (make use of list) for Research Intern/Exchange
program/Dual doctorate program
\end{enumerate}

\section{Semester 6}
\begin{enumerate}
\item \href{http://cdc.iitkgp.ac.in/p/foreign-training-opportunities}{Foreign training opportunities}
through CDC website
\item \href{https://ircell.iitkgp.ac.in/ftp/}{International relation cell} 
\item \href{http://cdc.iitkgp.ac.in/post/rs-opportunities}{RS opportunities} 
\item \href{https://international.iitkgp.ac.in/upcoming_opportunities/allevents/}{Upcoming opportunities} 
\item \href{https://ircell.iitkgp.ac.in/ftp/}{Foreign training program}
and login \href{https://ircell.iitkgp.ac.in/ftp/login/}{click here}
\item Semester away, Student exchange and Joint Doctoral Programs \href{https://international.iitkgp.ac.in/programs/}{check here}
\item Extension seminar for 3 to 4 year
\end{enumerate}

\section{Semester 7}
\begin{enumerate}
\item Keep working on PhD (8 hours)
\item Start working on new research area for postDoc according to list of
professor, list of university, alert from journal. (1 hour)
\item Start looking for foreign prof for post doc.
\end{enumerate}

\section{Semester 8}
\begin{enumerate}
\item Publish paper in collaborations
\item Extension seminar for 4-4.5 year
\item Look for people who can recommend (3 recommendations) you later. You
have published a collaborated paper. Contact collaborated Prof. 
\begin{itemize}
\item Ask concern people to \textbf{recommendation} you (make use of calender).
\end{itemize}
\item Start thinking about postdoc and search prof., university, fellowship.
\end{enumerate}

\section{Semester 9}
\begin{enumerate}
\item International conference
\begin{enumerate}
\item SERB funding
\item \href{https://international.iitkgp.ac.in/programs/}{Shri Gopal Rajgarhia International Program}
\item Other travel funding
\end{enumerate}
\item Start writing \textbf{Thesis}
\item Learn how to write research project proposal according to the interest
of the foreign prof.
\item Extension seminar for 4.5-5 year
\end{enumerate}

\section{Semester 10}
\begin{enumerate}
\item Finish your thesis
\item Start writing synopsis report
\item Complete synopsis report
\item Procedure for Synopsis and Thesis submission: find inside the repo.
\end{enumerate}

\part{PostDoc (University QS ranking < 500)}

\section{Introduction}
\begin{enumerate}
\item Research Intern after B. Tech. or M. Tech
\item Research fellowship for PhD
\item Postdoctoral Fellow (funding from institute)
\item Postdoctoral Fellow (student has its funding)
\end{enumerate}

\section{Type of fellowship}
\begin{enumerate}
\item Teaching fellowship
\item Individual fellowships 
\begin{enumerate}
\item Government funding
\begin{enumerate}
\item National Science Foundation in the U.S
\item Deutsche Forschungsgemeinschaft in Germany
\end{enumerate}
\item Institutional funding 
\end{enumerate}
\end{enumerate}

\subsection{List of foreign fellowships}
\begin{enumerate}
\item \href{https://academicpositions.com/career-advice/major-postdoc-fellowships}{Click here for list}
\item \href{https://asntech.github.io/postdoc-funding-schemes/}{GitHub}
\item \href{https://us.fulbrightonline.org/applicants/types-of-awards/study-research}{fulbrightonline},
\href{https://marie-sklodowska-curie-actions.ec.europa.eu/actions/postdoctoral-fellowships}{marie-curie},
\href{https://royalsociety.org/grants-schemes-awards/grants/newton-international/}{newton},
\href{https://www.humboldt-foundation.de/en/apply/sponsorship-programmes/humboldt-research-fellowship}{humboldt},
and others
\end{enumerate}

\subsection{List of Indian fellowships}
\begin{enumerate}
\item \href{https://www.noticebard.com/\%7C\%7CNoticeboard}{Noticeboard}
\end{enumerate}

\section{Where to search for postdoc job}

\subsection{Foreign}
\begin{enumerate}
\item \href{https://academicpositions.com/find-jobs}{academicpositions}
\item Canada \href{https://www.nserc-crsng.gc.ca/Students-Etudiants/PD-NP/index_eng.asp}{nserc-crsng}
\item USA \href{https://postdocinusa.com/postdoc-jobs/}{postdocinusa}
\item \href{https://academicjobsonline.org/ajo/jobs}{academicjobsonline},
\href{https://www.findapostdoc.com/}{findapostdoc}, \href{https://www.nature.com/naturecareers}{nature},
\href{https://jobs.sciencecareers.org/searchjobs/?Keywords=postdoc}{sciencecareers},
\href{https://www.postdocjobs.com/job/category/list?name=Engineering}{postdocjobs},
\href{https://www.eurosciencejobs.com/job_search/keyword/postdoc}{eurosciencejobs},
\href{https://www.postdoc.com/}{postdoc}
\end{enumerate}

\subsection{India}
\begin{itemize}
\item Go to IISC and then go to foreign
\end{itemize}

\section{Stage 1}
\begin{itemize}
\item Start building your network.
\item Start searching about a year before submitting your synopsis. 
\item Start your search for postdoc position by contacting the person(s)
who cited you. Reference: \href{https://www.lpl.arizona.edu/grad/guidebook/finding-funding/advice-and-tips-getting-jobpostdoc}{click here}
\item Follow prof. and university website. Here make use of Lists (including
fellowship). Reference: \href{https://www.brains-explained.com/how-to-apply-for-postdocs/}{click here}
\item Where to search:
\begin{enumerate}
\item Linkedin (follow targetted Prof. and university)
\item Twitter (follow targetted Prof. and university)
\item \href{https://imechanica.org/}{Imechanics} 
\item \href{https://academicpositions.com/find-jobs}{academicpositions.com},
\href{https://www.findapostdoc.com/}{findapostdoc} and \href{https://www.postdocjobs.com/}{postdocjobs}
\end{enumerate}
\end{itemize}

\section{Stage 2 }
\begin{itemize}
\item \textbf{Email}: \textquotedbl Dear XYZ, I saw that the XYZ postdoctoral
fellowship is being offered this fall. I'm trying finish my thesis
this spring, and was considering applying. I would certainly be interested
in working with you-{}-are you currently looking to take on any postdocs?
I'd be happy to discuss further if you're interested.\textquotedbl{}
\item \textbf{Cover Letter}: Emphasize how your work is relevant to the
goal of the postdoc, and reframe your work within their context of
the project. Briefly summarize your relevant qualifications for this
position. You want to convey how the lab or department will benefit
having you there, rather than how you will benefit from getting this
postdoc.
\item \textbf{Research Proposal}: Your research proposal explains the research
you would do during the postdoc. You should introduce the topic while
referring to past literature and point out the knowledge gaps in the
field that your project will fill. Then go through your timeline of
what you will accomplish (including publications) and when.
\item \textbf{Research statement}: Unlike the research proposal, this document
goes through your previous research experience and includes publications,
talks, conference presentations or posters that have come from it.
\item \textbf{CV}: \href{https://academicpositions.com/career-advice/how-to-write-a-professional-academic-cv}{click here}
\item \textbf{Recommendation}: Two or three good recommendation letters.
\end{itemize}
Reference: \href{https://academicpositions.com/career-advice/where-to-find-a-postdoc}{1},
\href{https://www.goldbio.com/articles/article/a-detailed-guide-to-your-postdoc-application-plus-printable-companion-checklist}{2},
\href{https://www.theguardian.com/higher-education-network/2015/feb/01/applying-for-a-postdoc-job-here-are-18-tips-for-a-successful-application}{3}

\section{Before thinking about postdoc}
\begin{itemize}
\item Published good and many papers.
\item Go to conference to make contact and find out what prof. are doing.
Make list of people.
\item Make CV
\item Make Web page
\end{itemize}

\section{During your last year}
\begin{itemize}
\item Contact your listed people by saying thank you and wish them on their
birth day or any specific occasional. 
\item Learn writing email about introducing yourself.
\item Write research proposal include figure.
\item Make ppt in addition to research proposal.
\item Previous/Current Phd research
\begin{itemize}
\item 2/3 page what you are doing now
\item Include attractive pictures. Graphical abstract will help here.
\item Highlight your skill relevant to publish papers. 
\end{itemize}
\item Research proposal for postdoc position
\begin{itemize}
\item What you will do in future related to post doc prof.
\item Define problem
\item How you will contribute to their lab.
\end{itemize}
\end{itemize}
Reference: \href{https://www.lpl.arizona.edu/grad/guidebook/finding-funding/advice-and-tips-getting-jobpostdoc}{Arizona}

\part{Academic position as faculty}
\begin{itemize}
\item Become member of professional boards/institutions. It is good to take
membership as it give wider connection for job and awards.
\end{itemize}

\section{Assistant Professor}
\begin{enumerate}
\item Outline your most significant contribution towards research.
\item How you as a faculty will contribute in the institute?
\item Present your vision for the department for the next five years.
\item Outline your proposed road map for teaching and research for the next
five years.
\item How would you create an innovative learning environment?
\end{enumerate}

\part{Nonacademic Job}
\begin{enumerate}
\item Linkedin, 
\item Subscribe to \href{https://www.facultyplus.com/}{facultyplus}, \href{https://www.facultyon.com/}{facultyon}
\item UPSC (only for legends)
\end{enumerate}
\begin{center}
Thank You \& Enjoy!
\par\end{center}
\end{document}
